\subsubsection{Pinia store}
A pinia store a globális változók használatára, illetve globális állapotkezelésre használatos, itt fogok olyan változókat és függvényeket definiálni, melyeket az alkalmazás számos komponensében alkalmazok. Ezzel elkerülhető a duplikálás, illetve segíti az alkalmazás betöltésekor párhuzamosan inicializálandó változók létrehozását, amelyekre csak bizonyos helyeken lesz szükség, ezzel csökkentve a töltőképernyők idejét és javítva a felhasználói élményt.
\begin{itemize}
    \item Változók
    \begin{itemize}
        \item $language$: A $Language$ nevű enum típus segítségével számontartja az aktuálisan választott nyelvet.
        \item $languageDict$: Egy szótár típus, mely előre definiált változókból kiválasztja az aktuális nyelvnek megfelelőt és tárolja.
        \item $model$: A Pythonban létrehozott és betanított neuronhálót tartalmazza.
        \item $modelPath$: A betöltendő neuronháló modell elérési útvonala.
        \item $loading$: Igaz, ha egy párhuzamos folyamat még éppen fut.
        \item $hunFlag$: A magyar zászló képének elérési útvonala.
        \item $engFlag$: Az Egyesült Királyság zászlaja képének elérési útvonala.
        \item $actualFlag$: Az aktuális nyelvhez tartozó zászló képének elérési útvonala.
    \end{itemize}
    \item Függvények
    \begin{itemize}
        \item $initializeStore$: Aszinkron függvény, betölti a neuronháló modellt a $model$ változóba.
        \item $changeLanguage$: Megváltoztatja az aktuális nyelvet.
        \item $getLanguageDictItem$: Az adott paraméterhez tartozó szöveget kiszedi a $languageDict$ szótárból.
        \item $startLoading$ és $endLoading$: A $loading$ flaget szabályozzák.
    \end{itemize}
\end{itemize}

\subsubsection{usePlay}
Úgynevezett composable függvény, amely olyan funkcionalitásokat foglal magába, melyeket több helyen is lehet alkalmazni, de nem globálisan, hanem mindig külön példányként. Ez a composable egy kapcsolatot teremt a $FreePlay.vue$, $NormalPlayGame.vue$ és a $DrawingPalette.vue$ komponensek között. Paraméterként a dependency injection elvnek megfelelően a $store$ változót kapja meg.
\begin{itemize}
    \item Változók
    \begin{itemize}
        \item $predId$: Az aktuális becslés azonosítója.
        \item $prediction$: A $predId$ - hoz tartozó szöveges megjelenítése a $getNameById$ segítő függvény alkalmazásával.
        \item $drawingPalette$: A $DrawingPalette.vue$ komponens referenciája.
        \item $predict$: Elvégzi a becslést.
    \end{itemize}
\end{itemize}

\subsubsection{App.vue}
Az alkalmazás fő komponense, jelen alkalmazásban csak egy $RouterView$ komponenst tartalmaz, ami az útvonal alapján dinamikusan fogja lerendelni az alkomponenseket.

\subsubsection{ProfileCreator.vue}
A $/profilecreator$ útvonalon érhető el, a felhasználó itt tudja megadni a felhasználónevét, majd tovább menni a következő oldalakra. Tartalmaz egy $InputText$ és egy $Button$ Primevue komponenset, előbbi egy bemeneti mezőt tartalmaz, ahova a felhasználó írni tud, utóbbi pedig elmenti a felhasználó által beírt nevet. Emellett tartalmaz egy zászlót, amellyel a nyelvet lehet átállítani.
\begin{itemize}
    \item Változók
    \begin{itemize}
        \item $playerName$: Itt fog eltárolódni a felhasználó által beírt szöveg, össze van kötve az $InputText$ komponenssel.
        \item $valid$: Ez fogja jelezni, hogy a felhasználó érvényes nevet adott - e meg és lehet - e tovább navigálni.
        \item $isError$: Jelzi, hogy érvénytelen szöveg van megadva, különbség a $valid$ változó között, hogy ez csak akkor igaz, ha hiba van a szövegben.
        \item $store$: HIVATKOZÁS
    \end{itemize}
    \item Függvények
    \begin{itemize}
        \item $checkErrors$: Leellenőrzi, hogy érvényes nevet adott - e meg a felhasználó, ennek függvényében állítja be a $valid$ és $isError$ változókat.
        \item $submit$: Elmenti a $playerName$ változó tartalmát a böngésző localstorage tárhelyében, a $userNameToken$ változóban, illetve tovább navigál a következő oldalra, nézve, hogy a felhasználó teljesítette - e már az oktatást.
    \end{itemize}
\end{itemize}

\subsubsection{Tutorial.vue}
Az oktató oldal komponense, az oktatás fő logikáját az $Owl.vue$ komponensben (HIVATKOZÁS) valósítottam meg, viszont itt a külső segítségek fognak megjelenni, mint például a $NavigationBar.vue$ (HIVATKOZÁS) és $DrawingPalette.vue$ (HIVATKOZÁS) komponensek, de itt lehet kihagyni is az oktatást egy gomb segítségével.
\begin{itemize}
    \item Változók
    \begin{itemize}
        \item $owl$: Az $Owl.vue$ komponens referenciája.
        \item $store$: HIVATKOZÁS
    \end{itemize}
    \item Függvények
    \begin{itemize}
        \item $onEnd$: Az $owl$ változó $end$ eseményének kezelésére használatos függvény, amely elmenti a böngésző localstorage tárhelyében egy $tutorialDoneToken$ változóban, hogy a felhasználó teljesítette az oktatást, majd a kezdőlapra navigál.
    \end{itemize}
\end{itemize}

\subsubsection{Home.vue}
Ez a komponens az alkalmazás kezdőoldalát teszi ki, amely csak navigálási célt szolgál. Tovább tudunk haladni a $FreePlay.vue$, $NormalPlay.vue$, illetve a $Tutorial.vue$ komponensek oldalára, illetve a nyelvet tudjuk változtatni.

\subsubsection{Main.vue}
A $FreePlay.vue$ és $NormalPlay.vue$ komponensek befoglaló komponense, egy $RouterView$ segítségével dinamikusan renderelődnek le, illetve a $NavigationBar.vue$ elemet mindkét komponenshez hozzácsatolja. Mindezek mellett, amennyiben a $store$ még nem fejezte be az inicializálást, akkor egy $ProgressSpinner$ Primevue komponens fog megjelenni, mint töltőképernyő.

\subsubsection{NormalPlay.vue}
A normál játék menüje, ahol a felhasználó beírhatja, hogy mennyi szót szeretne kapni a programtól a játék során. Egy $Dialog$ komponens magába foglalja a szám mezőt, illetve az elindító gombot. Amikor a játék befejeződött, ez a komponens fog kapni róla értesítést.
\begin{itemize}
    \item Változók
    \begin{itemize}
        \item $inMenu$: Jelzi, hogy éppen a menüben van a felhasználó, vagy éppen a játékban.
        \item $categoryNumber$: A választott kategóriák száma.
    \end{itemize}
    \item Függvények
    \begin{itemize}
        \item $start$: Elindítja a játékot, ezzel az $inMenu$ változót hamisra kapcsolja.
        \item $onRestart$: A $NormalPlayGame.vue$ komponens $restart$ esemény kezelésére szolgáló függvény, mely újraindítja a játékot lehetőséget adva a kategóriák számának újboli megadására.
    \end{itemize}
\end{itemize}

\subsubsection{FreePlay.vue}
A szabad játék komponense, ami csak a $DrawingPalette.vue$ rajzfelület komponenssel fog kommunikálni, illetve segítségül hívja a $usePlay$ composable függvényt, melyben minden szükséges funkcionalitás definiálva van, amely ennek a komponensnek a működéséhez szükséges.

\subsubsection{Owl.vue}
A $Tutorial.vue$ komponensben a bagoly szereplő beszédpanelét valósítja meg. Kettő publikus változóval és egy $end$ eseménnyel fog tudni kommunikálni a szülőkomponenssel az aktuális halad állapotával kapcsolatban. A beszédbuborék egy képre és egy szövegdobozra épül, amely a szöveg folyamatos "gepelését" szimulálja, így növelve az interaktivitást és a felhasználói élményt.

\begin{itemize}
    \item Változók
    \begin{itemize}
        \item $texts$: A megjelenítendő szövegek tömbje.
        \item $typedText$: Az aktuálisan megjelenített szöveg, amely karakterről karakterre épül fel.
        \item $typingSpeed$: A szimulált gépelés sebessége (milliszekundum/karakter).
        \item $typingInterval$: Az időzítő az automatikus szövegírás szimulációjához.
        \item $textIndex$: Az aktuálisan megjelenítendő szöveg indexe a $texts$ tömbben.
        \item $actualFullText$: Az aktuálisan teljesen megjelenítendő szöveg.
        \item $chatBubble$: A beszédbuborék HTML elemének referenciája.
        \item $fullText$: Jelzi, hogy a jelenlegi szöveg teljes egészében megjelenítésre került-e.
        \item $displayActionBar$: Jelzi, hogy megjelenjen-e az akció sáv.
        \item $displayDrawingPalette$: Jelzi, hogy megjelenjen-e a rajzolófelület.
    \end{itemize}
    \item Függvények
    \begin{itemize}
        \item $typeText$: Kezdeményezi a szöveg "gépelését" az aktuális indexű szöveggel.
        \item $showFullText$: Azonnal megjeleníti a jelenlegi szöveg teljes tartalmát, megszakítva a fokozatos megjelenítést.
        \item $onClick$: Ha a teljes szöveg már megjelenítésre került, továbblép a következő szövegre, vagy ha még nem, akkor azonnal megjeleníti a teljes szöveget.
        \item $scrollToBottom$: A beszédbuborék tartalmának görgetése az aljára.
        \item $shouldScroll$: Eldönti, hogy szükséges-e a tartalom aljára görgetni.
    \end{itemize}
\end{itemize}

\subsubsection{NormalPlayGame.vue}
\subsubsection{NavigationBar.vue}
Ez a komponens a navigációs sávot valósítja meg, amely lehetővé teszi a felhasználók számára, hogy gyorsan és egyszerűen navigáljanak az alkalmazás különböző részei között. A navigációs sáv tartalmazza a felhasználó nevét, valamint a nyelvváltáshoz szükséges zászló ikont. A komponens a $Menubar$ Primevue komponenst használja az elemek megjelenítésére. A navigációs gombok mellett elérhető a szavak listája, ami egy párbeszédpanelt nyit meg, amely megjeleníti az elérhető kategóriákat.
\begin{itemize}
    \item Változók
    \begin{itemize}
        \item $items$: A menüpontokat tartalmazó tömb, amely a navigációs sávban jelenik meg. Ez a tömb dinamikusan frissül a store állapotától függően.
        \item $isCategoriesDialogOpen$: Jelzi, hogy a kategóriák párbeszédpanelje meg van-e nyitva.
        \item $gridCols$: Meghatározza a kategóriák megjelenítéséhez használt grid oszlopainak számát a kijelző méretétől függően.
        \item $getUserName$: Kiolvassa a böngésző localstorage tárhelyéből a felhasználó megadott nevét.
    \end{itemize}
\end{itemize}
\subsubsection{DrawingPalette.vue}
Ez a komponens egy rajztáblát valósít meg, amely lehetővé teszi a felhasználók számára, hogy rajzoljanak a képernyőn egy adott területen belül, illetve rajzeszközöket válasszanak egy $CustomDock.vue$ dokkoló panelből.

\begin{itemize}
    \item Változók
    \begin{itemize}
        \item $canvas$: A rajztábla HTML canvas elemének referenciája.
        \item $ctx$: A canvas elem 2D rajzolási kontextusa.
        \item $drawing$: Jelzi, hogy a felhasználó éppen rajzol-e.
        \item $mode$: A rajzolási mód állapotát jelzi (igaz a rajzolásra, hamis a törlésre).
    \end{itemize}
    \item Függvények
    \begin{itemize}
        \item $predict$: Feldolgozza a $canvas$-ban készített rajzot, majd feldolgozza azt, amit a $store$-ban tárolt $model$ változónak átad, majd visszakapja a becslést.
        \item $clear$: Törli a rajztábla teljes tartalmát.
        \item $startDrawing$: A rajzolási folyamat kezdetét jelző függvény.
        \item $stopDrawing$: A rajzolási folyamat leállításához használt függvény.
        \item $draw$: Rajzolást végző függvény, amely követi az egér/érintés mozgását.
        \item $onModeChanged$: A rajzolási mód változásakor meghívódó függvény, amely lehetővé teszi a felhasználó számára, hogy váltson a rajzolás és a törlés között.
    \end{itemize}
\end{itemize}

\subsubsection{CustomDock.vue}

\subsection{Neuronháló tesztelés DNF}

\subsection{Webes felület tesztelés DNF}